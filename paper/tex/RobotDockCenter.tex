\documentclass{article}

% Packages for formatting
\usepackage{geometry}
\usepackage{setspace}
\usepackage{titlesec}

% Packages for figures and tables
\usepackage{graphicx}
\usepackage{caption}
\usepackage{subcaption}
\usepackage{booktabs}

% Packages for citations and references
\usepackage{natbib}
\usepackage{hyperref}

% Set document margins
\geometry{margin=1in}

% % Set line spacing
% \doublespacing

% Format section headings
\titleformat{\section}{\normalfont\Large\bfseries}{\thesection}{1em}{}
\titleformat{\subsection}{\normalfont\large\bfseries}{\thesubsection}{1em}{}

\begin{document}

% Title and author information
\title{\textbf{RobotDockCenter: A Novel and Supervised Learning Approach to Robot Docking Using Monocular Vision}}
\author{\textbf{Edward Ferrari}}
\date{\today}

\maketitle

\begin{abstract}
% Abstract of your paper
Robot docking is a critical task in autonomous systems. 
It allows a robotic entity to continue its designated workflow without the need for human intervention. 
Autonomous docking solutions have typically been centered on using and combining various sensors including LIDAR scanners and ultrasonic sensors, and even combining such sensors with cameras \citep{app131910675}. 
In the world of robotics, reinforcement learning has recently played a major role in further developing autonomy. 
In this paper, we propose an alternative approach to robot docking using monocular vision without the need of additional sensors or reinforcement learning. 
We use basic deep learning techniques to predict the next action based on a single input frame containing the target docking station.
Our approach is supervised and does not require any form of reward function or policy optimization.
\end{abstract}

\section{Introduction}

\section{Related Work}
The work of \citet{app131910675} provides a comprehensive review of existing docking methods.

\section{Methodology}
% Description of your proposed methodology or approach

\section{Experimental Results}
% Presentation and analysis of your experimental results

\section{Discussion}
% Discussion of your findings and their implications

\section{Conclusion}
% Summary of your work and future directions

% References
\bibliographystyle{plainnat}
\bibliography{references}

\end{document}
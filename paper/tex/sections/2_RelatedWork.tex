\section{Related Work}
The work of \citet{app131910675} provides a comprehensive overview of what robot docking usually consists of. That is, LIDAR-based mapping to calculate the position of the robot and use the surrounding environment such as walls to determine the next action the robot should take. In addition to LIDARs, the work uses cameras to detect the dock using a YOLO model, a similar aspect as to what our method consists of.

\citet{Deisenroth2012} revolves around the implications regarding reinforcement learning and how its implementation in low-power, low-cost systems leads to certain drawbacks that must be addressed. This relates to our work, as we take a different approach and use a supervised learning method to train our model, which aims to be less resource-intensive while still achieving good performance.

\citet{ameen2023optimizingdeeplearningmodels} explains the importance of optimizing models when inferencing on low-power, low-cost devices such as the Raspberry Pi. This is particularly important during our data processing phase, as we must deal with smaller numbers for ground truth values. Additionally, \citet{SINGH2020105524} further emphasizes the importance of data quality and the need for scaling data to ensure the model can learn effectively.

\citet{hoster2024usinggameenginesmachine} discusses the importance of game engines in simulating environments. Such environments can be used to create synthetic data, which is crucial for training models when real-world data is scarce or difficult to obtain. This aligns with our approach of using synthetic data gathered by Godot. Certain game engines, such as Unreal Engine, also provide tools that make scenes look more realistic, which can be beneficial for training models that need to generalize well to real-world scenarios. \citet{depth_anything_v2} also highlights the necessity of synthetic data in certain situations, while also pointing out that synthetic data is not always free of drawbacks depending on the task at hand. Synthetic data for our work was particularly useful in setting up an environment and testing the idea quickly rather than setting everything up in the real world, which would have taken much longer.

\citet{quilez:hal-01147332} showcases the use of infrared sensors combined with certain vision approaches such as QR code recognition in order to dock a robot. This is another instance of a robot docking method that uses sensors combined with vision, making it a hybrid approach. The paper makes it a clear goal to have a passive dock that does not require any active components, something that our dock also achieves.

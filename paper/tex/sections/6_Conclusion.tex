\section{Conclusion}
% Summary of your work and future directions
In this paper, we presented a novel approach to robot docking using monocular vision and deep learning techniques. Our methodology leverages a simulated environment created in Godot for data collection, followed by a robust training pipeline using PyTorch. The trained model predicts the necessary rotation and distance adjustments for the robot to dock accurately.

We demonstrated the effectiveness of our approach through extensive experiments, showing that our model can achieve high accuracy in predicting docking maneuvers. The integration of the inference script with a Flask server enables real-time predictions, facilitating seamless communication with the robot's control system.

Our results indicate that monocular vision, combined with deep learning, can be a viable solution for robot docking, eliminating the need for additional sensors or complex reinforcement learning algorithms. This approach not only simplifies the hardware requirements but also reduces the computational overhead, making it suitable for real-world applications.

Future work will focus on improving the model's robustness and generalizability by incorporating more diverse training data and exploring advanced neural network architectures. Additionally, we plan to integrate the system with physical robots to validate its performance in real-world scenarios.

Overall, our work contributes to the field of autonomous robotics by providing an efficient and scalable solution for robot docking, paving the way for more advanced and autonomous robotic systems.
